\subsection{Markedsbeskrivelse}

\subsubsection{Markedspotensiale}

Produktet vil bli brukt av firmaer og spesialister som allerede driver med radonmålinger. Det retter seg derfor mot et smalt marked, men fokuserer på å være det beste alternativet innen sin nisje. Mer spesifikt vil produktet selges til spesialister som bruker radonmålere og utfører tiltak mot radon i boliger hvor det er gjort en gjennomsnittsmåling av radon på over $100 Bq/m^3$. Produktet er ikke relevant for privatkunder som kun vil sjekke sin bolig, ettersom utsalgspris vil ligge på rundt 30.000 kr per enhet. Det er derfor heller ingen stor nytte å gjennomføre spørreundersøkelser om vårt produkt, da det kun selges til fagfolk. Det estimeres at det er marked for salg av rundt 20 produkter årlig i Norge, og det er derfor nødvendig å også rette seg mot et internasjonalt marked.

Statens Strålevern har uttalt på sine nettsider at de ønsker at alle boligeiere skal ta målinger, spesielt de som bor der det er mye alunskifer og er mer utsatt for høy radonkonsentrasjon.
Alle boligeiere burde foreta radonmåling hvert 5. år dersom man bor på et område der man er utsatt for mye radon, dette er viktig siden mye kan endre på radonkonsentrasjonen over tid, slik som utbygging i området, endring av ventiler osv.
Dette vil øke bruken av vårt produkt og markedet vårt vil øke, og etterhvert vil vi kanskje selge flere enn 20 produkter årlig i Norge.

\subsubsection{Internasjonalt markedspotensiale}

Det vil være lett å utvide seg til Sverige og Danmark, ettersom de har mye likt innen radonmarkedet. Norge og Scandinavia er enkle land å starte salget av produktet for å se om produktet funker som det skal og selger bra. Under testing av markedet er det viktig å snakke med andre land underveis for å høre om vårt produkt er noe de kan ha nytte av, er det etterspørsel? 
Borgevad har vært i kontakt med flere potensielle kunder, både i Norge, USA og Russland.
Tilbakemeldingene har vært ganske positive – han har blant annet blitt spurt om å få tilgang på en prototype før lansering.
Det virker lovende for å utvide internasjonalt, men man må ta hensyn til forskjellige regler i markedsføringen i de ulike landene som kan gjøre det litt komplisert. 
Større land som USA, Canada og Kina er vanskeligere å ekspandere seg til, da de ikke har et like fritt marked som i Skandinavia.
I USA er det rundt 400 lisensierte firmaer som retter seg mot radonmålere, i tillegg til et ukjent antall ulisensierte firmaer.
Markedet er derfor betydelig større enn i Norge.
Å komme seg inn på det amerikanske markedet medfører imidlertid en del implikasjoner.
I Norge kan hvem som helst drive med salg av radonmålere, mens man i USA må forholde seg til lisenser etc (dette må være mer spesifikt).

\subsubsection{Konkurranse}

De tilsvarende produktene (radonsnifferne) som er på markedet i dag bruker mye lenger tid enn dette produktet. For å finne kilden til den økte radonkonsentrasjonen i dag, estimeres tidsbruken til én time på 6-7 målinger. Dette produktet vil utføre de samme målingene på en brøkdel av tiden, noe som medfører en markant økning i effektiviteten. Dette produktet vil også være mer nøyaktig enn dagens radonmålere, da teknologien tillater en måling av alfapartikler direkte og nærmest umiddelbart. Man vil kunne måle alfapartiklene slik en metalldetektor fungerer, ved at man bare fører måleren langs overflaten, og man kan da effektivt lokalisere kilden. Med dette produktet vil tidsbruken minskes og kvaliteten økes. Arbeiderne, altså kundene av produktet, vil få en enklere jobb som kan utføres på mindre tid, og dermed høyere timelønn. Dette gjør produktet svært ettertraktet blant firmaer og spesialister som driver med diagnostisering av radonproblemer.

Vårt produkt henter målinger og sender dem ut på nett. Dette gjør det mulig å potensielt offentliggjøre en oversikt over målingene for allmenheten. Man kan bestille sporfilmer for å se om dette gjelder en selv hvis en ser at ens nabolag har flere hus med høye målinger. Dette vil åpne for at flere oppdager at de må gjøre tiltak i huset, og markedet kan da ekspanderes.

På markedet finnes det i Norge radonmålere som retter seg mot privat bruk, altså for å gjennomføre en gjennomsnittsmåling av bolingen din over tid. Det norskproduserte Canary Digital Radonmåler fra Corentium er et slikt produkt. Produktet som beskrives i denne rapporten er imidlertid det man trenger for å lokalisere kilden etter man har målt en høy gjennomsnittsmåling av radon. Selv om kundegruppen til disse produktene på nåværende tidspunkt ikke direkte overlapper, vil det være hensiktsmessig å vurdere de som en konkurrent dersom primærproduktet ekspanderes med en webapplikasjon for å holde styr på radonmålinger.

Et stort marked som kan få negativt utbytte av vårt produkt er boligmarkedet. Å vise på et kart hvor det er mye konsentrasjon av radon vil kunne endre planer i boligbygging ved at de i planleggingsprosessen må ta mer hensyn til radonmålingene, og dette er fordyrende. Planer som alt har startet vil få litt problemer med å selge hus som ligger i utsatte områder, og vil tape en god del penger på ukjøpte hus. Planer som er i begynnerfasen kan enten endre sted eller ta i bruk radonsperring på alle husene som skal bygges. I fra 2010 er det påbudt å bygge nye hus med radonsperre uansett om området er utsatt eller ikke. Eldre hus til salgs som viser seg å være i områder med mye radonkonsentrasjon kan bli vanskelig å selge, noe som kan føre til mange ukjøpte hus på boligmarkedet. Dette føre til at prisene på boligmarkedet går ned.
