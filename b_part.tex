\subsection{Del B}

\subsubsection{Marked/bransjebeskrivelse}

Hvordan kan markedet overordnet segmenteres/inndeles? Kvantifiser markedsstørrelsen. 
Hvem er kunden? Beskriv kundenytten 
Kort om inngangsbarrierer 
Kort om konkurrenter og substitutter 


Vårt produkt vil kun selges til spesialister og firmaer som bruker radonmålere og innretter tiltak mot radon i boliger som utlyser en høy gjennomsnitts måling av radon som ofte er over $100 Bq/m^3$. Produktet er ikke relevant for kunder som kun vil sjekke sin bolig ettersom det koster rundt 30.000kr for en enhet. Det er da heller ingen stor nytte å ha spørreundersøkelser om vårt produkt, da det kun selges til fagfolk. En vanlig husstand er nok ikke like begeistret for å investere 30.000kr for en radonmåler. Det fører til at det er rundt 20 kunder i året her i Norge som er våre kunder.  

Dagens produkter bruker lang tid, 6-7 målinger vil ta 1 time. Dette er grunnet at i dag oppstår det ofte problemer med å finne hvor kilden der radon kommer inn i byggningene ligger grunnet utvendige faktorer. Vårt produkt vil minske tidsbruken og øke kvaliteten, og kilden til radonutslipp kan lett finnes ved hjelp av ny teknologi lignende en metaldetektor. Arbeidere som går fra byggning til byggning for å senke radon i byggninger som har utlyst for høy måling av radon er de som vil bruke vårt produkt. Arbeiderne vil få en enklere, raskere jobb og høyere timelønn, som gjør produktet ettertraktet hos de fleste firmane og spesialistene som retter seg mot radonmåling. 

Vårt produkt henter målinger og sender de ut på nett, slik at alle kan få en god oversikt over hvor det er større risiko for å ha for høyt nivå av radon, og kan bestille sporfilmer for å se om dette gjelder en selv hvis en ser at deres nabolag har flere hus med for høye målinger. Dette vil øke bestilling av radonmålere og flere kan oppdage at de må gjøre tiltak i huset og vårt produkt vil bli mer brukt og nye kundervil kunne oppstå. 

Dette markedet gjelder kun Norge til nå, der det er lett å utvide seg til Scandinavia ettersom de har mye likt innen radonmarkedet. De større landene som U.S.A., Canada og Kina er vanskligere å ekspandere seg til, da de ikke har et like fritt marked som i Norge og Scandinavia når det gjelder radon. U.S.A. har rundt 400 lisensierte firmaer som retter seg mot radonmålere og et ukjent antall ulisensierte frimaer, noe som er betydelig mye større enn i Norge. I Norge vil det være mulig å kunne sende sensorene eller andre vesentlige deler på aparatet inn til reperasjon, eller eventuelt kan kunden bli tilsendt en ny del, hvis det er kun en av delene som er skadet. Det vil føre til at flere vil velge vårt produkt, da en ikke trenger å kjøpe et nytt aparat om noe skulle gå i stykker. På den måten vil aparatet vare lengre før det må byttes ut helt. 




