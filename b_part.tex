\subsection{Del B}

\subsubsection{Marked/bransjebeskrivelse}

Produktet vil bli brukt av firmaer og spesialister som allerede driver med radonmålinger. Det retter seg derfor mot et smalt marked, men fokuserer på å være det beste alternativet innen sin nisje. Mer spesifikt vil produktet selges til spesialister som bruker radonmålere og innretter tiltak mot radon i boliger som utlyser en gjennomsnittsmåling av radon på over $100 Bq/m^3$. Produktet er ikke relevant for privatkunder som kun vil sjekke sin bolig, ettersom utsalgspris vil ligge på rundt 30.000kr per enhet. Det er da heller ingen stor nytte å ha spørreundersøkelser om vårt produkt, da det kun selges til fagfolk. Det estimeres at det er marked for salg av rundt 20 produkter årlig i Norge, og det er derfor nødvendig å også rette seg mot et internasjonalt marked.

\subsubsection{Internasjonalt markedspotensiale}
Det vil være lett å utvide seg til Sverige og Danmark, ettersom de har mye likt innen radonmarkedet. Større land som USA, Canada og Kina er vanskeligere å ekspandere seg til, da de ikke har et like fritt marked som i Skandinavia. I USA er det rundt 400 lisensierte firmaer som retter seg mot radonmålere, i tillegg til et ukjent antall ulisensierte firmaer. Markedet er derfor betydelig mye større enn i Norge. Å komme seg inn på det amerikanske markedet, medfører imidlertid en del implikasjoner. I Norge kan hvem som helst drive med salg av radonmålere, mens man i USA må forholde seg til lisenser etc (dette må være mer spesifikt).

\subsubsection{Konkurranse}
De tilsvarende produktene som blir benyttet til radonmålinger og er på markedet i dag, bruker lengre tid enn det dette produktet vil være kapabel til. For å finne kilden til den økte radonkonsentrasjonen i dag, estimeres tidsbruken til én time på 6-7 målinger. Dette produktet vil utføre de samme målingene på bare 20 minutter, noe som medfører en markant økning i effektiviteten. Dette produktet vil også være mer nøyaktig enn dagens radonmålere, da teknologien tillater en måling av alfapartikler direkte og nærmest umiddelbart. Man vil kunne måle alfapartiklene slik en metalldetektor fungerer, ved at man bare fører måleren langs overflaten, og man kan da effektivt lokalisere kilden. Med dette produktet vil tidsbruken minskes og kvaliteten økes. Arbeiderne, altså kundene av produktet, vil få en enklere jobb som kan utføres på mindre tid, og dermed høyere timelønn. Dette gjør produktet svært ettertraktet blant firmaer og spesialister som driver med diagnostisering av radonproblemer.

Vårt produkt henter målinger og sender dem ut på nett. Dette gjør det mulig å potensielt offentliggjøre en oversikt over målingene for allmenheten, og kan bestille sporfilmer for å se om dette gjelder en selv hvis en ser at deres nabolag har flere hus med høye målinger. Dette vil åpne for at flere oppdager at de må gjøre tiltak i huset, og markedet kan da ekspanderes.

På markedet finnes det i Norge radonmålere som retter seg mot privat bruk, altså for å gjennomføre en gjennomsnittsmåling av bolingen din. Det norskproduserte Canary Digital Radonmåler fra Corentium er et slikt produkt. Produktet som beskrives i denne rapporten er imidlertid det man trenger for å lokalisere kilden etter man har målt en høy gjennomsnittsmåling av radon. Selv om kundegruppen til disse produktene på nåværende tidspunkt ikke direkte overlapper, vil det være hensiktsmessig å vurdere dem som en konkurrent dersom primærproduktet ekspanderes med en nettside eller programvare for å holde styr på radonmålinger.