\subsection{Markedsbeskrivelse}

\subsubsection{Markedspotensiale}

Produktet vil bli brukt av firmaer og spesialister som allerede driver med radonmålinger. Det retter seg derfor mot et smalt marked, men fokuserer på å være det beste alternativet innen sin nisje. Mer spesifikt vil produktet selges til spesialister som bruker radonmålere og utfører tiltak mot radon i boliger hvor det er gjort en gjennomsnittsmåling av radon på over $100 Bq/m^3$. Produktet er ikke relevant for privatkunder som kun vil sjekke sin bolig, ettersom utsalgspris vil ligge på rundt 30.000 kr per enhet. Det er derfor heller ingen stor nytte å gjennomføre spørreundersøkelser om produktet, da det kun selges til fagfolk. Det estimeres at det er marked for salg av rundt 20 produkter årlig i Norge, og det er derfor nødvendig å også rette seg mot et internasjonalt marked.

Statens Strålevern har uttalt på sine nettsider at de ønsker at alle boligeiere skal ta målinger, spesielt de som bor der det er mye alunskifer og er mer utsatt for høy radonkonsentrasjon.
Alle boligeiere burde foreta radonmåling hvert femte år dersom man bor på et område der man er utsatt for mye radon. Dette er viktig, da mye kan endre på radonkonsentrasjonen over tid, slik som utbygging i området og endring av ventiler.
Dette vil øke bruken av produktet, og markedet vil øke.

\subsubsection{Internasjonalt markedspotensiale}

Det vil være lett å utvide seg til Sverige og Danmark, ettersom de har mye likt innen radonmarkedet. Ved å starte salget av produktet i Skandinavia, får man sett om produktet fungerer som det skal, og man får også en indikasjon på hvor bra det selger. Under testing av markedet er det viktig å snakke med andre land underveis for å høre om produktet er noe de kan ha nytte av.
Borgevad har vært i kontakt med flere potensielle kunder, både i Norge, USA og Russland.
Tilbakemeldingene har vært positive – han har blant annet blitt spurt om å få tilgang på en prototype før lansering.
Større land som USA, Canada og Kina er vanskeligere å ekspandere til, da de ikke har et like fritt marked som i Skandinavia.
Det ser ut som produktet har stort internasjonalt potensiale, men man må her ta hensyn til lovverket i de ulike landene, noe som kan komplisere prosessen.

\subsubsection{Konkurranse}

De tilsvarende produktene (radonsnifferne) som er på markedet i dag, bruker lenger tid enn dette produktet. For å finne kilden til den økte radonkonsentrasjonen i dag, estimeres tidsbruken til én time på 6-7 målinger. Dette produktet vil utføre de samme målingene på rundt 20 minutt, noe som medfører en markant økning i effektiviteten. Dette produktet vil også være mer nøyaktig enn dagens radonmålere, da teknologien tillater en måling av alfapartikler direkte og nærmest umiddelbart. Man vil kunne utføre målingene på samme måte som en metalldetektor fungerer, ved at man fører måleren langs overflaten. Man vil da effektivt kunne lokalisere kilden. Med dette produktet vil tidsbruken minskes og kvaliteten økes. Arbeiderne, altså kundene av produktet, vil dermed få en enklere jobb som kan utføres på kortere tid. Dette gjør produktet svært ettertraktet blant firmaer og spesialister som driver med diagnostisering av radonproblemer.

Produktet henter målinger og sender dem ut på nett. Dette gjør det mulig å potensielt offentliggjøre en oversikt over målingene for allmenheten. Man kan bestille sporfilmer for å se om dette gjelder en selv hvis en ser at ens nabolag har flere hus med høye målinger. Dette vil åpne for at flere oppdager at de må gjøre tiltak i huset, og markedet kan da ekspanderes.

På markedet finnes det i Norge radonmålere som retter seg mot privat bruk, altså for å gjennomføre en gjennomsnittsmåling av boligen din over tid. Det norskproduserte Canary Digital Radonmåler fra Corentium er et slikt produkt. Produktet som beskrives i denne rapporten er imidlertid det man trenger for å lokalisere kilden etter man har målt en høy gjennomsnittsmåling av radon. Selv om kundegruppen til disse produktene på nåværende tidspunkt ikke direkte overlapper, vil det være hensiktsmessig å vurdere dem som en konkurrent dersom primærproduktet ekspanderes med en webapplikasjon for å holde styr på radonmålinger.

Et stort marked som kan få negativt utbytte av produktet er boligmarkedet.
Å vise på et kart hvor det er mye konsentrasjon av radon vil kunne endre planer i boligbygging ved at de i planleggingsprosessen må ta mer hensyn til radonmålingene, og dette er fordyrende.
Prosesser som allerede er i gang kan få økte problemer med å selge hus som ligger i utsatte områder, og vil tape en god del penger på ukjøpte hus.
Prosesser som er i begynnerfasen kan enten endre sted eller ta i bruk radonsperring på alle husene som skal bygges. Siden 2010 har det vært påbudt å bygge nye hus med radonsperre uansett om området er utsatt eller ikke. Eldre hus til salgs som viser seg å være i områder med mye radonkonsentrasjon kan bli vanskelig å selge, noe som kan føre til mange ukjøpte hus på boligmarkedet. Dette fører videre til at prisene på boligmarkedet går ned.
