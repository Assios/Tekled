\subsection{Organisering og økonomisk potensial}

\subsubsection{Organisering}

Den tenkte forretningsmodellen er å selge produktet til profesjonelle "radonjegere".
De betaler da en engangssum for én enhet.
Fordeler med dette kan være å ha høy inntjening tidlig, når man får solgt de første produktene.
Dette kan hjelpe på å skaffe kapital til å gjøre flere investeringer i utbedringer av produktet, og ekspansjon til flere markeder.
En ulempe kan være at det er vanskelig for mange kunder å gjøre en så stor engangsinvestering i et produkt, og at for mange ville muligens en modell med betaling over tid vært mer hensiktsmessig.
En slik måneds-/årspris kan være naturlig å kombinere med en supportavtale som inkluderer reparasjoner.
Her synes også et problem med modellen som baserer seg på en engangspris.
Dersom man får mange tilfeller av feil i produktet som må repareres, kan det bli dyrt å gjøre alle disse reparasjonene, med eventuelle tilbakekallinger av solgte enheter.
En løsning for å begrense omfanget av slike problemer, er å passe på at så mye som mulig av funksjonaliteten til enheter er implementert i programvare fremfor maskinvare.
Da er det lett å fikse uforutsette problemer ved å sende ut en programvareoppdatering.
Dersom enheten uansett har innebygd internett-tilkobling (for opplasting av data), er det trivielt å legge inn støtte for nedlasting av programvareoppdateringer i tillegg.

Teknologien til produktet stammer fra CERN, og de jobber sammen med personer fra Sintef med den videre utviklingen.
Det vil ikke være hensiktsmessig å bygge hele produktet selv fra grunnen av.
Selve mikrokontrolleren er det f.eks. mange firmaer som har spesialisert seg på å produsere, og det lønner seg derfor å kjøpe denne fra noen som leverer dette.
Det overordnede produkt-designet er noe man bør gjøre selv, men selve monteringen kan også med fordel outsources til noen som har erfaring og ekspertise på sammensetting av komponenter.

\textbf{Software:}
Uavhengig av om man går for en ren web-løsning, med direkte opplasting til internett fra enheten, eller om man velger en løsning hvor enheten sender data til en desktop-app, vil man trenge en programvareutvikler.
Men arbeidsmengden, og typen utvikling vil variere stort mellom disse to løsningene.
En desktop-løsning krever gjerne å lage applikasjoner for flere plattformer, slik som Windows, Mac og ev. Linux.
En ren web-løsning, derimot, kan være enklere å sette opp, og vil nok kreve mindre utviklingstid.

\subsubsection{Økonomisk potensial}

Produktet kan selges på markedet for rundt 30.000 NOK. Da kan man sitte igjen med omtrent 20.000 NOK.


\textbf{Investeringsbehov:}
For å unngå høye investeringskostnader i starten har de valgt å inngå avtaler med leverandører av komponenter til produktet. Ved å binde seg til komponenter fra en bestemt produsent, kan man kreve lavere priser i starten, når man har lite kapital. Dette lønner seg for produsenten fordi man da har bundet seg til å bruke dem i fremtiden, når man har midler til å betale full pris for komponentene.

\textbf{Investorer:}
Bedriften vi har snakket med er for tidlig i prosessen til å snakke med investorer, da det gjerne forventes at man kan vise til suksess i markedet før de er interesserte.
Veiledere de har snakket med om produktet er svært positive, og mener blant annet at det er en fordel at produktet inkluderer elementer av både teknologi og helse. Helse Midt-Norge er f.eks. en institusjon som deler ut midler til slike produkter, og kan være aktuell i fremtiden.
