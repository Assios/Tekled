%Oppgave 1: Innovasjon og organisasjonsstruktur 

\subsection{Eksempler på to organiseringsformer}

\subsubsection{Norsk Radiokontroll AS}

% delkapittel II og III fra Case NoRAS er mest relevant

Norsk Radiokontroll AS produserer RC-løsninger til industriformål.

Bedriften har to produksjonsavdelinger, én som står for mikrokontrollsystemene, og én som tar seg av plaststøping av styringsenheter.

Radio- og elektronikkomponenter er standard, og kjøpes inn fra eksterne produsenter.

De har også en egen markedsføringsenhet som besøker kunder.

Det er nærliggende å si at bedriften er organisert med en funksjonell struktur.
De to produksjonsavdelingene arbeider på hver sin del av det samme produktet, inndelt etter om de jobber med plaststøping eller mikrokontrollere.

Endring av bedriften:

Noen mener lederstilen er bra, problemet er at den tekniske kvaliteten er for dårlig.
Flere kunder har antydet dette.

De andre mener problemet er lederstilen til Elisabeth.
Hun er for teknisk opptatt, og detaljstyrer for mye.
Dette går utover hennes til overordnet styring av bedriften.

Plan etter møte:
Ansette ekspert på salg for å avlaste Elisabeth.
Utvide kompetanse på mikrosystemer
Opprette egen teknisk dokumentasjons-avdeling.

I april 1996 slår bedriften seg sammen med en plastbedrift fra Horten, Industriplast.
Noen måneder etter sammenslåingen ble de tre eierne av bedriften enige om å omstrukturere selskapet. Olav Heier, den tidligere sjefen fra Industriplast tok over som konsernsjef, slik at Elisabeth kunne tre over i en stilling som teknisk direktør.


\subsubsection{STOL AS}
% kanskje stol AS?

Nettverksstruktur:

I begynnelsen var bedriftens målgruppe privat sektor, hvor massesalg av samme proddukt var salgsstrategien deres. Etter hvert som de merket  at offentlig sektor også var interessant for dem, gikk de for flere varianter enn de to de allerede lagde. Organiseringen til bedriften var som når de hadde en mye mindre bedrift. Dette førte til at tredoblingen av ansatte førte til kommunikasjonsproblemer i produksjonsavdelingen.

Etter bortgangen til gründer Mikal Hansen, og den nye daglige lederen tok over, endret han organiseringen i produsjonsavdelingen.Han gjorde noen viktige tak i firmaet, for å øke omsetningen i bedriften samtidig med å skaffe bedre struktur og forenkle kommunikasjon innad. Han delte inn avdelingen i 3 mindre samlebånd og en monteringsavdeling, noe som økte effektiviteten. Samtidig sørget han for at de ansatte roterte på hvilke ansvarsområder de fikk, så de aldri hadde den samme oppgaven hele tiden, samt fleksibilitet ved sykemelding. Hovedsaklig ble kvaliteten på produktene bedre. Forskning på nye teknikker ble en viktig del for å tilfredsstille kvalitetskravene fra kundene.   

Stol AS skaffet seg en avtale med Mekk og Bøy, som var underleverandøren av metalldeler til bedrifter i området. Dette ble etter hvert en stor avtale for Mekk og Bøy og sto for over halvparten av omsetningen dems. Når Stol ekspanderte sin virksomhet til å gjelde flere produkter. Da måtte de skaffe leverandører for ull og skinn. for ull valgte de "dual-sourcing", altså to leverandører. or leveranse av skinn, valgte de å skaffe korte avtaler, hvor Stol AS skaffet avtaler med de med billigst materiale. \\


Mekk og bøy, avtale om leveranse av metalldeler til stolene.
Leverandør av ull, både ULLSPESIALISTEN og TYNSET-GARN.
Skinn - kortere kontrakter (tre måneder).
Samarbeid med SENG AS om å levere en bredere produktlinje (stoler, soveromsmøbler).

70-tallet var en periode med mange omorganiseringer, hvor de gikk fra avtale med to ull-leverandører til å kjøpe opp den ene av dem.

\subsection{Fordeler og ulemper med organiseringsmodellene}

\subsubsection{Norsk Radiokontroll AS}


\subsubsection{STOL AS}

Fordeler med å ha virksomheten distribuert utover flere firmaer er at de ulike delene av produksjonsprosessen blir separerte, slik at hver enhet kan fokusere på sin arbeidsoppgave.
Dette sparer mye overhead, som ville ha oppstått dersom alle disse funksjonene ble utført av én stor organisasjon.
For ull-leveransen, hvor de hadde avtaler med to firmaer, gjorde dette at de kunne ha en langsiktig avtale, men likevel ikke være avhengig av kun ett ledd.

Samarbeidet med SENG AS gjorde at de to firmaene til sammen kunne tilby et mye bredere produktassortement

Eksempel med MEKK OG BØY på slutten av 60-tallet, hvor avtalen ble formalisert, og de inngikk avtale om at nyansette hos MEKK OG BØY må ha to dager hos STOL AS som en del av opplæringen.
Dette for at ansatte hos MEKK OG BØY skulle vite mer om STOL AS sine behov.

Ulemper med en slik nettverksstruktur var at det ble mindre direkte samarbeid og påvirkning av de ulike bedriftenes interne prosesser.
