%Oppgave 1: Innovasjon og organisasjonsstruktur 

\subsection{Task A}

\subsubsection{Norsk Radiokontroll AS}

% delkapittel II og III fra Case NoRAS er mest relevant

Norsk Radiokontroll AS produserer RC-løsninger til industriformål.

Bedriften har to produksjonsavdelinger, én som står for mikrokontrollsystemene, og én som tar seg av plaststøping av styringsenheter.

Radio- og elektronikkomponenter er standard, og kjøpes inn fra eksterne produsenter.

De har også en egen markedsføringsenhet som besøker kunder.

Det er nærliggende å si at bedriften er organisert med en funksjonell struktur.
De to produksjonsavdelingene arbeider på hver sin del av det samme produktet, inndelt etter om de jobber med plaststøping eller mikrokontrollere.

Endring av bedriften:

Noen mener lederstilen er bra, problemet er at teknisk kvalitet er for dårlig.
Flere kunder har antydet dette.

De andre mener problemet er lederstilen til Elisabeth.
Hun er for teknisk opptatt, og detaljstyrer for mye.
Dette går utover hennes til overordnet styring av bedriften.

Plan etter møte:
Ansatte ekspert på salg for å avlaste Elisabeth.
Utvide kompetanse på mikrosystemer
Opprette egen teknisk dokumentasjons-avdeling.

I april 1996 slår bedriften seg sammen med en plastbedrift fra Horten, Industriplast.
Noen måneder etter sammenslåingen ble de tre eierne av bedriften enige om å omstrukturere selskapet. Olav Heier, den tidligere sjefen fra Industriplast tok over som konsernsjef, slik at Elisabeth kunne tre over i en stilling som teknisk direktør.


\subsubsection{En annen bedrift hvor man kan si noe om organisasjonen}
% kanskje stol AS?

\subsection{Task B}

