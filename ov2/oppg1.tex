%Oppgave 1: Innovasjon og organisasjonsstruktur 

\subsection{Eksempler på to organiseringsformer}

\subsubsection{Norsk Radiokontroll AS}

% delkapittel II og III fra Case NoRAS er mest relevant

Norsk Radiokontroll AS produserer RC-løsninger til industriformål.
Bedriften har to produksjonsavdelinger, én som står for mikrokontrollsystemene, og én som tar seg av plaststøping av styringsenheter.
Radio- og elektronikkomponenter er hyllevare, og kjøpes inn fra eksterne produsenter.

Bedriften var organisert med en funksjonell struktur, hvor de to produksjonsavdelingene arbeidet på hver sin del av det samme produktet, inndelt etter om de jobber med plaststøping eller mikrokontrollere.
Det var også en egen avdeling i bedriften som tar seg av markedsføring og salg.

I 1995 slet bedriften økonomisk, og flere ansatte mente at en stor del av problemet var lederstilen til Elisabeth. Hun var for teknisk fokusert, og detaljstyrte for mye. Dette gikk utover hennes kapasitet til overordnet styring av bedriften.

For å bøte på disse problemene bestemte bedriften i et ekstraordinært møte seg for å ansette en egen person til å ta ansvar for teknisk salg, og dermed avlaste en del av arbeidsoppgavene til Elisabeth. I tillegg vedtok de å omstrukturere bedriften til å ha en egen avdeling for teknisk dokumentasjon.

I april 1996 slo bedriften seg sammen med en plastbedrift fra Horten, Industriplast.
Noen måneder etter sammenslåingen ble de tre eierne av bedriften enige om å omstrukturere selskapet igjen.
Olav Heier, den tidligere sjefen fra Industriplast tok over som konsernsjef, slik at Elisabeth kunne tre over i en stilling som teknisk direktør.

\subsubsection{STOL AS}
% kanskje stol AS?

I begynnelsen var bedriftens målgruppe offentlig sektor, hvor massesalg av samme produkt var salgsstrategien deres.
Etter hvert som de merket at privat næringsliv, i tillegg til forbrukermarkedet, også var interessant for dem, gikk de for flere varianter enn de to de allerede lagde.
Dette førte til at bedriften vokste raskt, og raskt fikk en tredobling av ansatte, mens organiseringen til bedriften ikke endret seg noe særlig.
Dette skapte kommunikasjonsproblemer i produksjonsavdelingen.

Etter bortgangen til gründer Mikal Hansen, da den nye daglige lederen tok over, endret han organiseringen i produsjonsavdelingen.
Han gjorde noen viktige tiltak i firmaet, for å øke omsetningen i bedriften samtidig med å skaffe bedre struktur og forenkle kommunikasjon innad.
Han delte inn avdelingen i 3 mindre samlebånd og en monteringsavdeling, noe som økte effektiviteten. Samtidig sørget han for at de ansatte roterte på hvilke ansvarsområder de fikk, så de aldri hadde den samme oppgaven hele tiden, samt at dette ga mer fleksibilitet ved sykemelding.

STOL AS skaffet seg en langsiktig avtale med Mekk og Bøy, som var underleverandøren av metalldeler til bedrifter i området. Dette ble etter hvert en stor avtale for Mekk og Bøy og sto for over halvparten av omsetningen deres. Når Stol ekspanderte sin virksomhet til å gjelde flere produkter. Da måtte de skaffe leverandører for ull og skinn. for ull valgte de "dual-sourcing", altså to leverandører. or leveranse av skinn, valgte de å skaffe korte avtaler, hvor Stol AS skaffet avtaler med de med billigst materiale.

I 1997 inngikk STOL AS et samarbeid med SENG AS om å levere en bredere produktlinje, tilsammen både stoler, hyller/reoler og soveromsmøbler.

Samarbeidet mellom SENG AS, STOL AS og MEKK OG BØY representerer en nettverksstruktur.

\subsection{Fordeler og ulemper med organiseringsmodellene}

\subsubsection{Norsk Radiokontroll AS}

I 1995 ser vi at omsetningen synker og at mange mener dette skyldes Elisabeth og Johans lederstil. Spesifikt er det et problem at Elisabeth bruker for mye tid på å detaljstyre den tekniske mikrokontroller-avdelingen, noe som går på bekostning av tiden hun trenger til å styre firma.
Her kunne bedriften trengt et tydeligere vertikalt hierarki, slik at ikke toppledelsen blir blandet inn i detaljene i hvert av teamene.

\subsubsection{STOL AS}

Fordeler med å ha virksomheten distribuert utover flere firmaer er at de ulike delene av produksjonsprosessen blir separerte, slik at hver enhet kan fokusere på sin arbeidsoppgave.
Dette sparer mye overhead, som ville ha oppstått dersom alle disse funksjonene ble utført av én stor organisasjon.
For ull-leveransen, hvor de hadde avtaler med to firmaer, gjorde dette at de kunne ha en langsiktig avtale, men likevel ikke være avhengig av kun ett ledd.

Samarbeidet med SENG AS gjorde at de to firmaene til sammen kunne tilby et mye bredere produktassortement

Eksempel med MEKK OG BØY på slutten av 60-tallet, hvor avtalen ble formalisert, og de inngikk avtale om at nyansette hos MEKK OG BØY må ha to dager hos STOL AS som en del av opplæringen.
Dette for at ansatte hos MEKK OG BØY skulle vite mer om STOL AS sine behov.

Ulemper med en slik nettverksstruktur var at det ble mindre direkte samarbeid og påvirkning av de ulike bedriftenes interne prosesser.
