%Oppgave 3: Lærende organisasjoner
\subsection{Task A}
Da bedriften STOL AS ble laget, var organisasjonsstruktur enkel.
Gründer Mikal Hansen hadde en idé og holdt denne godt for seg selv.
Bedriften hadde en avtale med MEKK OG BØY AS om leveranse av metalldeler til produktet. Denne avtalen var aldri formalisert og baserte seg mye på muntlige taler.
På dette tidspunktet var det taus kunnskap som skapte verdi for bedriften.
Da bedriften hadde så få ansatte, klarte de seg med en liten administrasjon og behøvde ikke eksplisitt kunnskap. Alle visste hva alle gjorde, og det var lett å få gitt riktige beskjeder.

Da bedriften deretter fikk en enorm ekspansjon fra 40 til 150 ansatte, merket man at den tause kunnskapen skapte problemer. Da det før var enkelt å få beskjeder frem og alle visste hva alle gjorde, var det nå vanskelig å få kunnskap frem og mange var usikre.
Dette er et klassisk problem når et selskap ekspanderer.
Uten eksplisitt kunnskap hviler man på menneskelig kommunikasjon for å tilordne seg kunnskap, og dette skalerer ikke.
Heller ikke teknologien som ble brukt var eksplisitt. Den eksisterte bare i hodet på gründeren.

Da gründeren døde, ble dette en ny utfordring for selskapet. Teknologien som ble brukt måtte gjenoppfinnes.
En slik prosess er både kostbar og dyr for en bedrift, og tilsier at kunnskap som er vital for en bedrift burde være eksplisitt.

Da det kom nye ledere inn i firmaet, endret kunnskapskulturen seg. Den tause kunnskapen ble med en gang eksternalisert til eksplisitt kunnskap.
Alle ansatte ble bedt om å skaffe seg kunnskap innen flere områder i bedriften for å øke dekningsgraden av kunnskap hos de ansatte.
Ut fra teksten virker det som om de ansatte hadde taus kunnskap om maskinene, så det kan ha vært spesialmaskiner til bedriftens formål.
Det ble også igangsatt opplæring hos underleverandører for å heve det generelle kunnskapsnivået i alle bedriftens ledd.

Ved videre ekspansjon ble den samme kunnskapskulturen holdt.

Etter en stund ble markedet endret. Billigforretninger som IKEA spiste av markedet, og bedriften falt mellom to stoler. Da var styret og lederne uenige, så lederne gikk.
De nye lederne som kom inn dro nytte av at det allerede var mye kunnskap hos alle ansatte, både taus og eksplisitt.
Da bedriften endret retning til å satse på miljøvennlig plast, baserte de seg mye på eksplisitt kunnskap og deling av kunnskap.

\subsection{Task B}
Da STOL AS ble opprettet, var det få virkemidler i bruk for å dele kunnskap i bedriften. 
Gründer Mikal Hansen var hovedårsaken til dette.
Han holdt all kunnskap om trelaminater for seg selv. STOL AS kom til å nærme seg konkurs da Hansen døde.
All teknologien han hadde funnet opp for bedriften var dårlig notert. 
Her kunne STOL AS ha fått Hansen til å dele kunnskapen sin med andre i bedriften ved å ha konferansemøter og kurs. I tillegg kunne Hansen ha benyttet seg av midler som ville ha gjort teknologien hans mer lestbar for etterkommere.

Siden STOL AS hadde liten ledelse og få ansatte, kunne de ansatte lett få tak i kunnskap om hvem som gjorde hva i bedriften. Deling av kunnskap og kompetanse gikk for det meste via kommunikasjon mellom ansatte i begynnelsen.

Etter hvert som bedriften ekspanderte, ble det vanlig at ansatte kun fokuserte på én maskin eller ett område.
Dette viste seg å bli en ulempe når det gjaldt sykemelding og nyansatte. 
Hvis alle som arbeidet på samme maskin ble syk, kunne ingen andre ansatte ha tatt over den aktuelle maksinen, og produkjonen ville ha stanset.
Dette kunne STOL AS ha forhindret ved å lære flere nyansatte forskjellige maskiner og områder ved hjelp av kurs og orientering.
Senere ble det vedtatt at alle ansatte skulle lære seg tre maskiner hver.

Flere endringer kom da Torstein Varhei overtok STOL AS.
Ved å dele bedriften inn i avdelinger med avdelingsledere, ble det lettere å dele kompetanse og kunnskap innad i bedriften og avdelingene.
Avdelingslederne kunne holde møter og kurs innen ny teknologi for å sørge for at alt i sin avdeling gikk som det skulle.
Kurs og opplæring ble gitt til MEKK OG BØY og andre heleide bedrifter, slik at de visste hva produktene deres ble brukt til.

Etter krakket var halve bedriften kuttet ned og underbedrifter solgt. Det ble mer fokus på kvalitet enn kvantitet.
Her ville det ha egnet seg for STOL AS å ha kurs innen ny teknologi, mote, trend og lignende regelmessig for å holde seg gående og oppnå profitt.
Gruppeaktiviteter blant ansatte med ulik kompetanse og kunnskap kunne bidra til å frembringe nye teknologier og produkter til STOL AS.
