%Oppgave 3: Lærende organisasjoner
\subsection{Task A}
Da bedriften STOL AS ble skapt var det en enkelt struktur.
Gründer Mikal Hansen hadde en idé og holdt denne godt for seg selv.
Bedriften hadde an avtale med MEKK OG BØY AS om leveranse av metalldeler til produktet. Denne avtalen var aldri formalisert og baserte seg mye på muntlige taler.
På dette tidspunktet var det taus kunnskap som skapte verdi for bedriften.
Når bedriften hadde så få ansatte, klarte de seg med en liten administrasjon og behøvde ikke eksplisitt kunnskap. Alle visste hva alle gjorde og det var lett å få beskjeder frem til der de skulle.

Når bedriften deretter får en enorm ekspansjon fra 40 til 150 ansatte merker man at den tause kunnskapen skaper problemer. Da det før var enkelt å få beskjeder frem og alle visste hva alle gjorde, er det nå vanskelig å få kunnskap frem og mange er usikre.
Dette er et klassisk problem når et selskap går fra oppstartsfase til ekspensasjon.
Uten eksplisitt kunnskap hviler man på menneskelig kommunikasjon for å tilordne seg kunnskap og dette skalerer ikke.
Heller ikke teknologien som ble brukt var eksplisitt, men eksisterte bare i hodet på gründeren.

Da gründeren døde ble dette en ny utfordring for selskapet. Teknologien som ble brukt måtte gjenoppfinnes.
Dette er både kostbart og dyrt for en bedrift og tilsier at kunnskap som er vitalt for en bedrift burde være eksplisitt.

Da det kom nye ledere inn i firmaet endret stilen seg. Den tause kunnskapen ble med én gang eksternalisert til eksplisitt kunnskap.
Alle ansatte ble bedt om å skaffe seg kunnskap til flere områder i bedriften for å øke dekningsgraden av kunnskap hos de ansatte.
Ut fra teksten virker det som om det var taus kunnskap om maskinene, så det kan ha vært spesialmaskiner til bedriftens formål.
Det ble også igangsatt opplæring av underleverandører for å heve det generelle kunnskapsnivået i alle bedriftens ledd.

Ved vidre ekspansasjon den samme kunnskapskulturen er holdt.

Etter en stund er markedet endret. Billigforetninger som IKEA spiser av markedet og bedriften faller mellom to stoler. Her er styret og lederne uenig, så lederne går.
De nye lederne som kom inn dro nytte av at det allerede var mye kunnskap hos alle ansatte, både taus og eksplisitt.
Når bedriften endret retning til å satse på miljøvennlig plast baseres mye på eksplisitt kunnskap og deling av kunnskap.

\subsection{Task B}
Da STOL AS ble opprettet var det få virkemidler i bruk for å dele kunnskap i bedriften. 
Gründer Mikal Hansen var hovedgrunnen til dette, han holdt all kunnskap om trelaminater for seg selv. STOL AS kom til å nærme seg konkurs da Hansen døde og all teknologien han hadde funnet opp for bedriften var dårlig notert. 
Her kunne STOL AS ha fått Hansen til å dele kunnskapen sin med andre i bedriften ved å ha konferansemøter og kurs. I tillegg kunne Hansen benyttet seg av midler som ville gjort teknologien hans mer lestbart for etterkommere.

Ved å ha en liten ledelse på 5 og få ansatte kunne de som jobbet for STOL AS lett få tak i kunnskap om hvem som gjør hva i bedriften. Deling av kunnskap og kompetanse gikk for det meste via kommunikasjon mellom ansatte på starten.

Etter hvert som bedriften ekspanderte og fikk nye samarbeidspartnere ble det vanlig at ansatte kun fokuserte på en maskin/område. Dette viste seg å bli en ulempe når det gjaldt sykemelding og nyansatte. 
Hvis alle på en maskin ble syk kunne ingen andre ansatte tatt over den aktuelle maksinen, og produkjonen ville ha stanset opp. 
Dette kunne STOL AS forhindret ved å lære flere ansatte forskjellige maskiner/områder ved hjelp av kurs og orientering når man starter. 
Senere ble det vedtatt at alle ansatte skulle lære seg 3 maksiner hver.



