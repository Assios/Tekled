\subsection{Taxi-app}

Vi har kommet opp med en forretningsidé som vil gjøre det enklere og mer effektivt både for kunden og for taxiselskapene å finne hverandre. Ideen går ut på at alle taxiselskapene listes i applikasjonen vår, slik at man ved behov kan kontakte drosjer i nærområdet.

Taxiselskapene kan også, om det ønskes, installere en GPS-enhet i drosjene sine, slik at brukeren kan få informasjon om drosjer i nærområdet. Brukeren kan så bestille drosjen ved et tastetrykk, og drosjen vil da få opp lokasjonen til kunden. Dersom drosjen godtar bestillingen, får kunden beskjed om at drosjen er på vei, og da får også kunden opp avstands/tidsestimat på drosjen.

\subsection{Radonmåler}

Denne forretningsideen er en forbedret versjon av et allerede eksisterende produkt: radonmåler.

Problemet med tradisjonelle radonmålere er at de er tungvinte å bruke.
Man må gå til radonmåleren og enten manuelt lese av verdier på et lite display eller plugge enheten i en datamaskin og laste ned dataene.
I tillegg bruker måleren lang tid på å registrere radonkonsentrasjon, noe som gjør den lite egnet til feilsøking i sanntid.
Radonmåleren vi ser for oss laster, derimot jevnlig opp data direkte til skyen, slik at måledataene kan aksesseres på en webside eller i en app.
Dataene kan presenteres som grafer og sammenlignes med nåværende forskrifter.
Tilgang til dataene for en bestemt måler krever innlogging.
Den største fordelen med radonmåleren vil være at den kan flyttes rundt i en bygning for å måle radonkonsentrasjon på ulike posisjoner. Dette gjør at den raskt kan identifisere hvor radon lekker inn i boligen, slik at man kan utbedre disse feilene.
Den kan ha GPS og andre sensorer som automatisk registrerer posisjon.

Dagens forskrifter for radon i nye bygninger sier at radonkonsentrasjonen må være under en viss verdi.
Derfor er behovet stort for å automatisere og forenkle utføring av disse målingene.
