\subsection{Del A}

\subsection{Produkt/tjenestekonsept}

Det nye produktet kommer av en ny teknologi fra Cern. Teknologien gjør at det er enklere og raskere å måle nivået av alfapartikler. Som sagt tidligere kunne en prøve ta rundt 10 min, men med dette apparatet skal man få resultatene i faktisk sanntid. Dette fører til en mye mer effektiv måleprosess, hvor man vil bruke mindre av arbeidstiden til å utføre målingene. Samtidig skal denne måleren være mye mer pålitelig en de som nå er på markedet. Man skal ikke trenge å få problemer med apparatet, som at man må lufte det.	

Tanken med dette produktet er å ha en webapplikasjon til produktet. Når man utfører en måling skal det være mulig å legge inn måledataen trådløst inn på en datamaskin, som gjør at det blir enklere og aksessere og tolke informasjonen. Samtidig vil det gjøre at man lettere har informasjonen tilgjengelig til senere om det trengs, det vil være lettere å oppdatere hvordan konsentrasjonsnivået av radongass er og man vil ha mulighet til å utnytte måledataene på en helt annen måte. Det skal senere være mulig å vise informasjonen på et kart. Tanken er at når måledataene blir lagt inn på computeren, vil kartet bli oppdatert automatisk, med informasjon som gps-lokasjon, dato og strålemengden. En annen mulighet som også er mulig, er at selve måleren laster opp måledataen rett opp til nettapplikasjonen, som igjen bruker informasjonen til å direkte oppdatere og legge til den nye informasjonen.
