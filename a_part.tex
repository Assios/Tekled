\subsection{Del A}

\subsubsection{Problem}

Luft som inneholder radongass avgir alfastråling. Når man puster inn denne luften, blir bronkiene og lungene eksponert for denne helseskadelige strålingen. I verste konsekvens er strålingen fatal. Radongass medvirker til at cirka 300 nordmenn dør av lungekreft hvert år. Internasjonalt er dette tallet mye høyere.

For å finne ut om et oppholdsrom har for høy konsentrasjon av radon, må man måle luften over en periode på minst to måneder. Dette bør gjøres mellom midten av oktober til midten av april, fordi da er radonkonsentrasjonen mest stabil. Myndighetenes tiltaksgrense er $100 Bq/m^3$. Hvis en måling viser en høyere verdi enn dette, må man gjennomføre tiltak som reduserer radonnivået i rommet.

Det er en utfordring å diagnostisere radonproblemet. Grunnen til høyt radonnivå i et oppholdsrom er ofte en bygningsteknisk svakhet. Når gulv og vegger nær bakken har små sprekker og utettheter, kan det sive inn jordluft som inneholder radon. For å finne utettheter, bruker man vanligvis en av følgende metoder:
\begin{itemize}
	\item Setter på et undertrykk i huset og går rundt med et elektronisk måleinstrument kalt sniffer. Dette måleinstrumentet har et rør som suger inn luft på et punkt og måler radonkonsentrasjonen i denne luften.
	\item Setter på en røykmaskin og ser hvor røyken siver inn. Denne metoden gir ikke noe måletall, men man ser hvor en eventuell sprekk befinner seg.
\end{itemize}

Dagens sniffere er ikke optimale:
\begin{itemize}
	\item Det tar en stund (opptil 10 minutter) å få en måling
	\item De har en tendens til å være ustabile. Noen ganger må man ta de med ut i ren luft en stund før man fortsetter å måle.
	\item De måler ikke alfapartiklene som radongassen avgir, men datterproduktene til radon.
\end{itemize}

Noen av dagens radonmålere kan kobles til en PC med spesialprogramvare for å overføre måledata. Etterpå kan disse lastes opp til en sentral database. Denne prosessen er tungvindt, og kan forenkles for å spare tid.

Et tredje behov vi ser er at det i dag ikke finnes noen karttjeneste som viser radonmålinger på et detaljert nivå. Statens strålevern har en grov figur som viser gjennomsnitlig radonnivå per kommune, basert på tilfeldige målinger. Problemet med denne figuren er at den viser gjennomsnitlig radonnivå per kommune, ikke per adresse. Den viser også bare 4 forskjellige radonnivåer, og blir ikke automatisk oppdatert. Hvis man har et slikt kart, vil hvem som helst kunne se radonmålinger i nærheten, eller ved en bolig man ønsker å kjøpe. Dagens løsning er at man kan kontakte Statens Strålevern, og så får man etterhvert en rapport hvis man har rett til innsyn.



\subsubsection{Produkt/tjenestekonsept}

Det nye produktet kommer av en ny teknologi fra Cern. Teknologien gjør at det er enklere og raskere å måle nivået av alfapartikler. Som sagt tidligere kunne en prøve ta rundt 10 min, men med dette apparatet skal man få resultatene i faktisk sanntid. Dette fører til en mye mer effektiv måleprosess, hvor man vil bruke mindre av arbeidstiden til å utføre målingene. Samtidig skal denne måleren være mye mer pålitelig en de som nå er på markedet. Man skal ikke trenge å få problemer med apparatet, som at man må lufte det.	

Tanken med dette produktet er å ha en webapplikasjon til produktet. Når man utfører en måling skal det være mulig å legge inn måledataen trådløst inn på en datamaskin, som gjør at det blir enklere og aksessere og tolke informasjonen. Samtidig vil det gjøre at man lettere har informasjonen tilgjengelig til senere om det trengs, det vil være lettere å oppdatere hvordan konsentrasjonsnivået av radongass er og man vil ha mulighet til å utnytte måledataene på en helt annen måte. Det skal senere være mulig å vise informasjonen på et kart. Tanken er at når måledataene blir lagt inn på computeren, vil kartet bli oppdatert automatisk, med informasjon som gps-lokasjon, dato og strålemengden. En annen mulighet som også er mulig, er at selve måleren laster opp måledataen rett opp til nettapplikasjonen, som igjen bruker informasjonen til å direkte oppdatere og legge til den nye informasjonen.

\subsubsection{Utfordringer med produktet}

Det har tidligere vært noen få problemer med den nye måleren. Når de brukte den til forsøk i Cern, kunne det være bakgrunnsstøy og fått målinger, når det egentlig ikke var noe å måle. Det har vært jobbet en del med å finne ut av hva grunnen til dette var. I etterkant viste det seg at grunnen til dette var støy på selve strømkilden, så dette har allerede blitt fikset og er ikke lengre noe problem.

For å få produsert måleren, skal det kjøpes inn deler i utlandet, de skal sendes til Norge og bli satt sammen her. 
